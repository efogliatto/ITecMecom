\documentclass[a4paper,11pt,twoside]{MECOM}

%%%%%%%%%%%%%%%%%%%%%%%%%%%%%%%%%%%%%%%%%%%%%%%%%%%%%%%%%%%%%%%%%%%%%%%%%%%
%               Parametros principales del documento                      %
%%%%%%%%%%%%%%%%%%%%%%%%%%%%%%%%%%%%%%%%%%%%%%%%%%%%%%%%%%%%%%%%%%%%%%%%%%%

% Titulo
\titulo{Implementaci\'on de la clase MECOM para generaci\'on de informes t\'ecnicos en \LaTeX{}}

% Autores
\autores{E. O. Fogliatto}{}

% Revisores
\revisores{}{}{}

% Revision de calidad
\calidad{A. Gonz\'alez}

% Aprobacion
\aprobacion{E. Dari}

%Objetivo
\objetivo{El presente documento tiene como objetivo mostrar el uso de la clase MECOM.cls para gererar informes t\'ecnicos.}

% Alcance
\alcance{Su utilizaci\'on est\'a restringida al Departamento de Mec\'anica Computacional. GOOD SAVE \LaTeX{}!}

% Numero de informe tecnico
\numeroIT{XX/20XX}

% Metadatos para pdf
\hypersetup{
    pdfauthor={Ezequiel O. Fogliatto},
    pdftitle={Implementaci\'on de la clase MECOM para generaci\'on de informes t\'ecnicos en \LaTeX{}},
    pdfkeywords={\LaTeX{}, Informes T\'ecnicos, Mecom},
    pdfcreator={CNEA},
    pdfsubject={Departamento de Mec\'anica Computacional, IN-ATN40MC-XX/20XX}    
    }
    
% Autores de revisiones
\definechangesauthor[name={Johann Sebastian Mastropiero}, color=blue]{mas}






\begin{document}

    % Creacion de la caratula
    \portada
    
    % Creacion del indice
    \tableofcontents   
    
    % Comienzo del desarrollo del documento
    \printnomenclature[2cm]


        
    \section{Uso de MECOM.cls}
    La clase {\bf MECOM.cls} est\'a basada en {\bf article} \cite{hefferon_minutes_2005}, y est\'a pensada para escribir informes t\'ecnicos de Mecom. Se recomienda consultar el archivo ITec\textunderscore Mecom.tex para ver la implementaci\'on de secciones como portada, \'indice, abreviaturas, referencias, etc.
    \nomenclature[A]{Mecom}{Mec\'anica Computacional}



    
    \subsection{Portada}
        El comando {\bf \textbackslash portada} genera la portada en una p\'agina nueva. Para completarla es necesario llenar los siguientes campos.
        \begin{itemize}
            \item titulo
            \item autores
            \item revisores
            \item calidad
            \item aprobacion
            \item objetivo
            \item alcance
            \item numeroIT (s\'olo la parte final, es decir n\'umero/a\~no)
        \end{itemize}
        \par
        Es necesario conservar el logo de CNEA\nomenclature[A]{CNEA}{Comisi\'on Nacional de Energ\'ia At\'omica} para poder compilar el documento. Junto con este documento se adjunta un archivo \emph{Makefile} para facilitar la compilaci\'on completa (\'indices, abreviaturas, y referencias).
        \par
        Los tama\~nos de letra de la portada se encuentran fijos en Arial 11pt. Al modificar el tama\~no global, la portada no se modifica.


        
    \subsection{\'Indice}
        Se redefini\'o la estructura de {\bf \textbackslash tableofcontents} para colocar el \'indice en una p\'agina nueva, con la palabra {\bf \'INDICE} centrada y los t\'itulos de secciones y subsecciones en tama\~no \textbackslash large.
        


        
    \subsection{Abreviaturas}
        Uso del paquete {\bf nomencl}. Los s\'imbolos est\'an alineados y tienen sangr\'ia, y se defini\'o el nombre de los grupos Acr\'onimos y S\'imbolos para diferenciar, por ejemplo, CNEA de $\epsilon$.\nomenclature[N]{$\epsilon$}{Letra griega para nomenclatura}


        
    \subsection{Im\'agenes y tablas}
    Se hace uso del paquete {\bf caption} para dar formato a los ep\'igrafes, y las referencias generan links por defecto en el pdf. Por ejemplo, ver como queda la \fig{fg:Logo_CNEA}
    \begin{figure}[ht]
        \centering
        \includegraphics[width=0.5\textwidth]{Logo_CNEA}
        \caption{Ejemplo de uso de figuras}
        \label{fg:Logo_CNEA}
    \end{figure}
    
    \par
    Por defecto, tambi\'en se carga el paquete {\bf subcaption}, de modo que se pueden insertar subfiguras como en la \fig{fg:subfiguras}. Para facilitar la uniformidad en las referencias, pueden usarse las macros {\bf \textbackslash fig, \textbackslash figs y \textbackslash tb}, tambi\'en definidas en MECOM.
    
    \begin{figure}[ht]
        \centering
        \begin{subfigure}[t]{0.45\textwidth}
            \centering
            \includegraphics[width=0.9\textwidth]{Logo_CNEA}
            \caption{Subfigura}
            \label{fg:sub1}
        \end{subfigure}
        \begin{subfigure}[t]{0.45\textwidth}
            \centering
            \includegraphics[width=0.9\textwidth]{Logo_CNEA}
            \caption{Subfigura}
            \label{fg:sub2}
        \end{subfigure} 
        \caption{Uso de subfiguras con {\bf subcaption}}
        \label{fg:subfiguras}       
    \end{figure}    
    
    
    \par
    Para usar las tablas se definieron los anchos de l\'inea con el paquete {\bf booktabs}, y pueden quedar como la \tb{tab:prueba}.
    \begin{table}[ht]
        \centering
        \begin{tabular}{c c}
            \toprule
            \bf Procesador & \bf Caracter\'istica \\
            \midrule
            \LaTeX{} & Excelente \\
            OpenOffice & Porquer\'ia\\
            \bottomrule
        \end{tabular}
        \caption{Ejemplo de uso de tablas}
        \label{tab:prueba}
    \end{table}    
    


    
    \subsection{Referencias}
    Para generar este documento se us\'o el estilo {\bf asmems4} con Bibtex, y se incluye el archivo asmems4.bst para facilitar la primera compilaci\'on. El usuario de la clase puede emplear el formato que considere m\'as adecuado.
    \par
    La clase MECOM hace uso del paquete {\bf tocbibind}, el cual aporta flexibilidad en la generaci\'on del t\'itulo e inclusi\'on en el \'indice. Por defecto, el t\'itulo de la secci\'on {\bf Bibliograf\'ia y referencias} se encuentra numerado, aunque este efecto puede cambiarse eliminando la opci\'on {\bf numbib}.
    
    
    
    \subsection{Control de cambios}
    En esta primera versi\'on, se sugiere el uso del paquete {\bf changes} para realizar revisiones (hasta que se defina una alternativa mejor). Este paquete permite, entre otras cosas, incorporar anotaciones como en \replaced[id=mas]{LibreOffice}{Microsoft Word}. Se pueden incluir autores, colores, lista de cambios, etc.
    \par
    La utilizaci\'on de este paquete en realidad no implica llevar a cabo un control de las versiones, sino remarcar las correcciones y/o extensiones que puedan incorporar los revisores. Si se usa la clase con la opci\'on \emph{vfinal}, la compilaci\'on produce un pdf sin estas marcas.
    
    
    
    \subsection{Formato general}
    El resto de las caracter\'isticas generales del documento, como sangr\'ia, interlineado, formato de t\'itulo de secciones, encabezados, etc., se encuentran definidas dentro de la clase, utilizando valores similares a los empleados en el template con formato .docx. La clase MECOM est\'a comentada, de forma que puede cambiarse f\'acilmente.
    
    
    
    
    \section{Creaci\'on y uso de repositorios con \emph{git} (red interna de mecom)}
    La herramienta de control de versiones \emph{git} puede usarse en la elaboraci\'on del documento, siendo especialmente \'util en la edici\'on por parte de m\'ultiples usuarios (por ejemplo, correcci\'on y revisi\'on). A continuaci\'on se incluyen algunas instrucciones simples para el uso de git.
    
    \subsection*{Repositorios remotos}
    En primer lugar es necesario contar con el repositorio local, el cual podr\'ia crearse siguiendo los siguientes pasos    
    \begin{verbatim}
        cd mi_proyecto/
        git init
        git add *
        git commit -m "Commit inicial"
    \end{verbatim}
    En este caso, \emph{mi\textunderscore proyecto/} es el directorio donde se elabora el informe, \emph{git init} inicializa el proyecto para \emph{git} creando el directorio \emph{.git/}, \emph{git add *} agrega todos los archivos al 'sttaged area', y \emph{git commit} genera el commit inicial. Para m\'as informaci\'on, ¡leer la documentaci\'on de \emph{git}!.
    \par
    Si lo que se desea es compartir este proyecto, entonces puede generarse un repositorio remoto, donde las personas que trabajen sobre el mismo puedan hacer sus \emph{pull, push y fetch}. Para ello, es necesario crear e inicializar un directorio para el repositorio remoto (en este ejemplo, el creador del repositorio remoto es el mismo que el creador del local):
    \begin{verbatim}
        cd
        mkdir -p ubicacion_de_preferencia/mi_proyecto.git
        cd ubicacion_de_preferencia/mi_proyecto.git
        git --bare init
    \end{verbatim} 
    Dentro de este repositorio remoto pueden agregarse las opciones    
    \begin{verbatim}
        git config core.sharedrepository 1
        git config receive.denyNonFastforwards true
    \end{verbatim}
    donde la primera obliga a git a mantener los permisos de lectura y escritura, mientras que la segunda obliga al usuario a hacer \emph{merge} en la m\'aquina local, y luego subir los cambios. Una vez listo el repositorio remoto, volver al repositorio local y agregar un \emph{remote}
    \begin{verbatim}
        git remote add origin ~/ubicacion_de_preferencia/mi_proyecto.git
    \end{verbatim}
    de modo que, ejecutando la siguiente instrucci\'on dentro del repositorio local
    \begin{verbatim}
        git push origin master
    \end{verbatim}
    se origina el primer \emph{push} al repositorio remoto y \'este queda listo para ser usado.
    \par
    Una vez que se cuenta con el repositorio remoto, {\bf con permiso de lectura y escritura}, un usuario B puede clonar el proyecto creado por el usuario A. Si se ejecuta, por ejemplo
    \begin{verbatim}
        cd Documents
        git clone /users/usuarioA/ubicacion_de_preferencia/mi_proyecto.git
    \end{verbatim}
    entonces el usuario B est\'a clonando el proyecto remoto \emph{mi\textunderscore proyecto.git} creado por el usuario A, dentro del directorio Documents. Cuando el usuario B cuente con el proyecto, puede editarlo, hacer sus commits, y luego subirlo al repositorio remoto:
    \begin{verbatim}
        git push origin master
    \end{verbatim}
    De modo que el usuario A puede (luego de verificar que existen cambios en el repositorio remoto) actualizar su repositorio local, por ejemplo con
    \begin{verbatim}
        git pull origin master
    \end{verbatim}    

    \par
    Con este procedimiento, dos (o m\'as) usuarios pueden editar el documento manteniendo un control de cambios. 
    \par
    En el ejemplo mostrado, los dos usuarios trabajan desde sus cuentas de mecom. Sin embargo, uno puede agregar \emph{remotes} usando protocolos como \emph{ssh, http, https o git}. Por ejemplo, si se desea clonar un proyecto en una m\'aquina que cuenta con acceso remoto al cab, podr\'ia hacerse
    \begin{verbatim}
        git clone ssh://usuarioB_casa@localhost:/users/usuarioA/../mi_proyecto.git
    \end{verbatim}
    o bien, para crear un repositorio en forma remota
    \begin{verbatim}
        git remote add origin ssh://usuarioA_casa@localhost:/../mi_proyecto.git
    \end{verbatim}      
    
    
    
    
    
    
    \section{Conclusiones}
    Ya no hay excusas para no documentar el trabajo usando informes t\'ecnicos.\\    
    
    
    \bibliographystyle{asmems4}
    \bibliography{MiBiblioteca}
    
    
    \newpage
    \section{Prueba de m\'argenes} \blindtext[11]    


\end{document}
